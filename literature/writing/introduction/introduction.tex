% Part about UniProtKb
With protein and nucleotide sequencing getting cheaper every year, 
large-scale sequencing efforts are increasing the coverage of complete proteomes.
The amount of protein sequences in UniProtKB has been growing exponentially.
As on the moment this paragraph was written, 
UniprotKB contained a total of 181,252,700 protein sequences. 
562,253 were reviewed by expert curators and part of the UniProtKB/Swiss-Prot database.
180,690,447 unreviewed sequences are part of the UniprotKB/TrEMBL database. 
That means that experimental information about functionality or biophysical features is only available for about 0.3 percent of the proteins.
For 99.7 percent, only the protein sequence is known.
% Relation between protein function and primary sequence
Local and distant interactions along this the structure guide the folding process,
thereby determining structure , dynamics and function.
Essentially, all of the biophysical features are encoded within the primary sequence.
However, predicting and extracting these features, given only the protein sequence, has proven to be a challenging task,
and still a topic of ongoing research.

% More specific about dynamics
Traditionally, protein function was thought to be a direct result of protein structure,
But in recent years it has become clear that the structure provides only part of the picture.
Other biophysical features, such as the propensity toward dynamics, disorder, and early folding also seem to play a fundamental role for protein functionality.
A new generation of machine learning based predictors that only need a single protein sequence as input
are being developed at the Bio2Byte research lab at the Interuniversity Institute of Bioinformatics in Brussels.
These predictors are trained on in-solution characteristics that highlight the importance of local interactions.
Two such predictors are Dynamine, which predicts backbone an sidechain dynamics,
and EFoldMine, which predicts how likely amino acids are involved in early folding events.
Not only can these predictors predict biophysical features without relying on prior structural knowledge,
the predictions are also very fast,
making them scalable for computational high-throughput methods.

% Protein localization and signal peptides
Cells are organised into different subcellular compartments including
the cytoplasm, the periplasm, the cell wall, and the nucleosome.
Signal peptides are short polypeptides at the N-terminal end of proteins that act as zip-code within the cell,
providing information to the cell how to sort the proteome over the different compartments.
These peptides typically have a length between 25 and 30 residues.
In general, signal peptides display low sequence similarity,
but there is a strong conservation of biophysical properties.
While signal peptides play an important role in protein translocation,
they are on themselves insufficient to guarantee this process.
A study (\cite{orfanoudaki2017}) has shown that cytoplasmic proteins can be distinguished from the secreted with an accuracy up to 95 percent using only features present in the mature domain.
One prominent difference seems to be that secreted proteins delay the folding process 
and display an increased flexibility and dynamics.
The reason for this difference, and what biophysical effect signal peptides have on these characteristics remains unclear.

% What is unknown and what was researched
In this master thesis,
the biophysical features of cytoplasmic and periplasmic proteins were compared on an extensive Gram-negative Bacteria protein dataset.
In addition, the effect of signal peptides on these features was examined.
The dataset was extracted from the public database UniProtKB by making use of their annotation system.
Biophysical predictions were acquired with state of the art machine learning based predictors that require only a single protein sequence as input,
making this method applicable for the full protein sequence space.

