Points in a multiple dimensional space can be fitted by a line.
The best fit can be defined as one that minimizes the the average squared distance from the points to the line.
The basis vector describing this line is called the first principal component.
The basis vector that describes the next best fitting line, orthogonal to the first is second principal component,
a third best fitting line orthogonal to both previous forms the third principal component and so on.

UniRep vectors were first scaled to a mean of 0 and standard deviation of 1 with the \textit{preprocessing} module of the \textit{sklearn} python library.
Principal components were generated with using the \textit{decomposition} module.
It uses the method of Minka,
which uses a covariance matrix.
The main goal of performing the principal component analysis in this master thesis was to visualize the 64 dimensional UniRep vectors on a 2-dimensional graph.
Visualization was done wit \textit{matplotlib}.
