UniProtKB provides a service that maps entries from one database to another.
Those databases can be internal (UniProtKB, UniRef, UniParc, etc.),
but also external (PDB, RefSeq nucleotide, Entrez Gene, etc.). 
The mapping service is both available by web interface and by programmatic access as REST API.

Code of the implementation can be found in the Appendix, section \textit{uniprotMapping.py}.
It implemented as a function that can be used in pipelines,
it accepts 7 parameters:
\textbf{
fileName,
query,
From,
To,
Format,
Columns,
and
OutputDir.
}
The \textbf{query} accepts a set of identifiers in string format, which can be separated by spaces or newline symbols.
The \textbf{From} and \textbf{To} parameters accept respectively from which database the query is mapped and to which one.
With the \textbf{Format} parameter,  different output formats can be chosen.
The parameters \textbf{fileName} and \textbf{OutputDir} indicate under what name and where the file will be stored.

It should be noted that it is better to limit the amount of identifiers per request.
Generally it is a good idea not to map more than a couple of thousand identifiers at once,
as more will tend the request to fail, 
but not always with giving an error.
Especially in the context of a pipeline, 
nothing is as frustrating than starting it to run overnight,
only to see that it failed at one of the first steps before any work was really done.
To avoid this, it is more robust to put 30 requests of 1,000 identifiers, than one request of 30,000.
