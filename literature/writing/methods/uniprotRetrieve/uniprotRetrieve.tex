UniProtKB provides a REST API to retrieve data via queries similar to how text search works on the website interface.
The API utilises the native functionality of HTTP GET requests.
The idea is that all of the necessary information is encoded within the URL.
Subsequently a request is made to the UniProtKB server, 
which collects the data from the database, 
restructures it to the requested formatn
and sends the information back to the user.

The URL follows a certain format (shown below in the frame),
it starts with a fixed base 
(\url{https://www.uniprot.org/uniprot/?}), 
complemented  by a set of 'parameter=value' pairs.
The order of the parameters is of no importance,
and not all parameters have to be provided.
The server will fall back to default values in that case.
The \textbf{query} has a similar format to what is used by the website text search interface,
but spaces are replaced by '+' symbols.
There are different  \textbf{formats} as which the data can be downloaded:
\textit{
html,
tab,
xls, 
fasta,
gff,
txt,
xml,
rdf,
list,
and rss}.
The \textbf{columns} parameter indicates which fields should be provided in the \textit{tab} or \textit{xls} format.
The \textbf{include} parameters means different things, depending on what format was requested.
There is the possibility to Gzip the data, which is provided by the \textbf{compress} parameter.
Finally, if the user only wants N results,
it can be passed with the \textbf{limit} parameter.
By default, the counts starts from the first entry, 
but this can be changed with the \textbf{offset} parameter.


~\begin{tcolorbox}
	\url{
	https://www.uniprot.org/uniprot/?
	query=<<QUERY>>
	&
	format=<<FORMAT>>
	&
	columns=<<COLUMNS>>
	&
	include=<<yes|no>>
	&
	compress=<<yes|no>>
	&
	limit=<<LIMIT>>
	& 
	offset=<<OFFSET>>
	}
 ~\end{tcolorbox}

Implementation of the REST API was done in Python (code in Appendix section \textit{uniprotRetrieve.py}).
Datasets resulting from a query searches can easily comprise multiple gigabytes,
even more than what is available in RAM memory.
Therefore, the \textbf{requests} (\cite{reitz2012}) library was chosen to handle the HTTP GET requests,
because it has the option to retrieve data as a stream of chunks.
This allows the data to be stored on the hard disk directly, without putting a load on memory.
Sometimes the UniProtKB server is overloaded by requests and will not give anything back.
As this could break a pipeline,
the implementation was made more robust to these types of failures by
designing it to retry up to 10 attempts to do a request.
Each attempt, the program will wait an increasingly amount of time before putting a new request. 
Servers often ignore request that come in to frequently from the same origin.
Also special care was taken so that the \textbf{query} parameter can be provided in the same manner as when you would use the text search interface on the website. 
This way, a query can be gradually build up with the query builder, 
and subsequently copied as argument for in a pipeline.
