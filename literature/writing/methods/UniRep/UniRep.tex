\cite{alley2019} provides a github repository (\url{https://github.com/churchlab/UniRep}) with a tutorial on how to use the mLSTM-RNN.
Three different architectures are provided, 
generating a UniRep vector of length 64, 128, or 1900.
A function is provided to generate a UniRep vector for a given protein sequence.

The UniProtKB query retrieval REST API was utilised to retrieve periplasmic and cytoplasmic proteins from the class of Gammaproteobacteria.
Sequences were clustered by cd-hit to and identity of 50 percent.
UniRep vectors of length 64 were generated for the representatives of each cluster.
For the periplasmic proteins, a vector was generated based on a proteins sequence containing the signal peptide,
and another vector excluding the signal peptide.

Classifiers were constructed based on different machine learning methods:
multiple linear regression,
decision tree,
random forest,
Support Vector machine,
and neural network.
Each of these was implemented from scratch in python,
code can be found on \url{https://github.com/ancnudde/Artificial_intelligence}.
The implementation of these classifiers was a group project for with the following collaborators:
Anthony Cnudde,
Jan Van den Schilden,
Guillaume Buisson-Chavot, 
and Jessica Ody.

Performance is given by following equation:
~\begin{equation*}
	accuracy = \frac{Correctly\ predicted\ in\ test\ set}{Correctly\ and\ wrongly\ predicted\ in\ test\ set}
~\end{equation*}
