When retrieving proteins with the query retrieval REST API, 
it can be specified which features should be present (or absent).
For the cytoplasm proteins, only proteins having an annotation of being located in the cytoplasm or cytosol were kept.
As additional check, no signal peptides could be present.
For the periplasmic proteins, the query asked for proteins with an annotation being located in the periplasm.
In addition, a signal peptides should be present.

Only Gram-negative Bacteria contain a periplasm. 
However, Gram-negative Bacteria are not a monophylogenetic taxonomic group.
Therefore, multiple datasets were generated on different taxonomic levels:
\textit{
Enterobacteriaceae,
Enterobacterales,
Gammaproteobaceria,
Proteobacteria,
}
and \textit{Bacteria}.
As example, a typical search query is shown below.
Proteins were downloaded in \textbf{tab}-format with the following \textbf{columns}:
\textit{
	id,
	organism,
	ec,
	sequence
}

~\begin{tcolorbox}
	\textbf{Cytoplasmic proteins:}

	QUERY="taxonomy:Bacteria (locations:(location:cytoplasm) OR locations:(location:cytosol)) NOT annotation:(type:signal)"
	\\
	\\
	\textbf{Periplasmic proteins:}

	QUERY="taxonomy:Bacteria locations:(location:periplasm) annotation:(type:signal)"
~\end{tcolorbox}

In addition to previous searches, 
it would be interesting to know for which twins there were PDB structures available.
Therefore, a similar query search was performed as described above, but with the extra limitation
(shown below).

~\begin{tcolorbox}
	\textbf{Cytoplasmic proteins:}

	QUERY="taxonomy:Bacteria database:(type:pdb) (locations:(location:cytoplasm) OR locations:(location:cytosol)) NOT annotation:(type:signal)"
	\\
	\\
	\textbf{Periplasmic proteins:}

	QUERY="taxonomy:Bacteria locations:(location:periplasm) annotation:(type:signal) database:(type:pdb)"
~\end{tcolorbox}

 The query on all Bacteria resulted in \textbf{7,772,586 cytoplasmic} and 
 \textbf{171,333 periplasmic} proteins. 
 It should be noted that the phylum of Proteobacteria already contained 169,484 of the periplasmic proteins,
making it the largest Gram-negative group.
