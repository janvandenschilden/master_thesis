To search for twins, the command line \textit{blastp} tool was used from the \textbf{ncbi-blast+} package.
The general syntax for the tool is given below.

~\begin{tcolorbox}
	blastp -query $<<$QUERY.FASTA$>>$ -db $<<$DATABASE$>>$ -evalue $<<$EVALUE$>>$
~\end{tcolorbox}

A database can be generated from a fasta file with following command.

~\begin{tcolorbox}
	makeblastdb -in $<<$FASTA$>>$ -dbtype 'prot' -out $<<$DATABASE$>>$
~\end{tcolorbox}

Custom Python wrapper function were written for those two commands
(for code see Appendex section \textit{generateDatabases.py} and \textit{runBlast.py}).

For each organism, a database was generated using the periplasm \textit{fasta} file.
Against this database the, cytoplasm \textit{fasta} file was searched for homology detection.
As threshold, an e-value of 1e-10 was chosen.
Finally, a custom parser (for code appendix section \textit{extractTwinsFromBlast.py})
extracted the twins for the \textit{blast} file format to a simple \textit{tab} file.
Within this file, each row represents a twin, one columns contains the identifier of the cytiplasmic protein, one the identifier of the periplasmic protein.
In total, \textbf{31,165 twins} were found,
spread over \textbf{780 different organisms}.
