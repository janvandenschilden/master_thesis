The biophysical properties of signal peptides are highly conserved through evolution and present in both prokaryotes and eukaryotes.
Their presence is a clear indication that the protein in question is translocated to another subcellular location of extracellular environment.
Many predictors were designed to detect these signal peptides and their cleavage sites(\cite{armenteros2019}),
and others have included this information to predict subcellular location (\cite{jaramillo2014}).
Various machine learning techniques have been utilised for this problem including:
weight matrix methods,
hidden Markov models,
support vectors machines,
and neural networks.
For example,
SignalP 5.0 (\cite{armenteros2019}) uses a deep neural network to predict  signal peptides across all domains of life,
and in prokaryotic organisms it distinguishes between the three types (Sec/SPI; Sec/SPII; Tat/SPI).
