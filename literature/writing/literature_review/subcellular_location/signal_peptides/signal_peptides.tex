Signal peptides have been elaborately studied since they were first hypothesized by Blobel and Sabatini (\cite{blobel1971}).
They are short polypeptides at the N-terminal end of proteins, the part that is generated first by ribosomes,
and carry information for the protein target location or secretion.
They occur in all domains of life and have been of interest both industrial and scientific fields,
ranging from industrial recombinant protein production and biotherapeutics to immunization and laboratory techniques (\cite{owji2018}).
The gram positive model organism \textit{Bacillus subtilis} contains for about 50 percent of the extracellular proteins a typical signal peptide (\cite{humphery2006})
For the gram negative model organism \textit{Escherichia coli}, this is more than 90 percent (\cite{rusch2007}),
and also eukaryotes contain a significant amount of secretory proteins with signal peptide (\cite{kanapin2003}).
Even though multiple studies examined signal peptide function and structure,
some features remain to be elucidated.

