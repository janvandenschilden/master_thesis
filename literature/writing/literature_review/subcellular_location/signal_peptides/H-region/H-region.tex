The hydrophobic H-region has generally a length between 7 and 15 residues and has multiple functions.
The hydrophobic residues have propensity toward forming an $\alpha$-helix.
They have a membrane spanning role,
thereby helping to orient the signal peptide.
In addition, the amino acid composition helps determining the secretory pathway, 
and protein processing (e.g. N-linked glycosylation).
In general, an increase of hydrophobicity results in higher protein translocation.
However, jamming of the translocase has been observed when hydrophobicity is increased to much,
but low hydrophobicity can slow down the translocation rate
(\cite{mordkovich2015}).
Getting the optimal balance for secretion has long been an subject of interest in protein engineering.
In Gram-negative Bacteria,
increasing the hydrophobicity augments inner membrane secretion,
but decreases outer membrane translocation.
The H-region also determines which secretory pathway the protein will end up using (Sec, SRP or Tat).
Highly hydrophobicity stabilizes the SP-SRP interactions, thus favoring the SRP pathway.
Signal peptides destined to the Tat and Sec pathway are less hydrophobic.
The role of hydrophobicity for the choice of pathway seems be more important in prokaryotes than eukaryotes,
and more significant in Gram-negative bacteria than the Gram-positive
(\cite{owji2018}).
Leucines are the most dominant residue in H-regions.
Even though the exact amino acid composition differ significantly throughout species,
conserved H-motifs are observed within the same species
(\cite{duffy2010}).
This means that signal peptides are not always interchangeable between species,
even if they have the same hydrophobicity.
Finally, H-regions often have an helix breaker present.
This enables the signal peptide to form a hairpin structure and insert itself more easily in the membrane.
Helix breakers are proline, glycine or serine residues and can be found at any position except on the edges of the H-region.
