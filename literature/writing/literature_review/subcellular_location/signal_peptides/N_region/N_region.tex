The N-region is generally around 5 residues long.
It consists mainly out of basic, positively charged side sidechains,
namely lysine, arginine and histidine.
This positive charge has multiple functions.
For a start, the positive charges interact with the negatively charged phosphate groups in the lipid layer of the plasmamembrane
(\cite{von1990}).
The balance of positive charge along the signal peptide will dictate the orientation,
thereby favoring or preventing translocation.
By having a charge difference between the C-region and N-region in favor of the latter,
the orientation of the cleavage site is more easily accessible to the SPase active site.
Additionally, the positively charged residues interact with the phosphate backbone of the SRP complex or SecB chaperone.
This interactions delays the folding process and keep the protein in transport competent state
(\cite{hardy1991}).
Interestingly, not only the amino acids composition itself,
but also translation efficiency play an important role in translocation.
In about 40 percent of the \textit{Escherichia coli} signal peptides,
the second codon is AAA (Lys).
Substitution of the codon to the lower translation-initiation-rate codon AAG has been observed to prevent secretion
(\cite{zalucki2007}).

