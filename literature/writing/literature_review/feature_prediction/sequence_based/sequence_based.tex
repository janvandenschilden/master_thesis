While it is still not possible to predict the behaviour of the whole protein from sequence, 
there exist machine learning based predictors that can predict certain characteristics. 
A new generation of predictors, developed at the VUB, 
use in-solution characteristics of proteins and include information about intrinsically disordered regions, setting them apart om structure-based methods. 
The predictions are based on statistical information and highlight the importance of local interactions in a sequence between amino acids close to each other in the sequence, and the relative positions of different residues, 
e.g. a tyrosine could hypothetically increase the backbone dynamics when 3 amino acids apart from a valine, 
but reduce it when 7 amino acids apart. 
These predictions are very fast, which makes them scalable for computational high-throughput methods. 
DynaMine (\cite{cilia2013}) predicts the backbone dynamics of a protein. 
The predictor is trained on NMR data of proteins in solutions. 
Disordered regions are identified with similar accuracy as compared to the best other predictors, 
but without relying on structural information or prior disorder knowledge.
It even shows great potential in distinguishing between folded domains, molten globules, disordered linkers and pre-structured binding motifs of different sizes. 
Other DynaMine flavors, for sidechain dynamics and secondary structure propensities, 
were developed to create EFoldMine (\cite{raimondi2017}). 
This method predicts which amino acids are likely involved in early folding events, as detected by hydrogen deuterium exchange data from NMR pulsed labelling experiments. 
Predicted early folding residues are frequently those with a lot of interactions in the folded structure and they also display evolutionary covariation. 
Early folding events are likely defined by intrinsic local interactions and could have lasting effect in subsequent states. 
