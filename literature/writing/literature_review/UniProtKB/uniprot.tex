With protein and nucleotide sequencing getting cheaper every year,
large-scale sequencing efforts are increasing the coverage of complete proteomes for a wider range of organisms than ever before. 
Additionally, the improvement of experimental techniques provide deeper information about structure and function of individual proteins.
The UniProt knowledgedatabase (UniProtKB) contains a wealth protein sequences and associated information.
For over half a million of these protein sequences,
expert curators have extracted detailed annotations from literature.
The UniProt Knowledgebase stores these reviewed proteins as UniProtKB/Swiss-Prot entries.
An automated pipeline, including a rule-based system, annotates the majority of unreviewed proteins.
These are stored as UniProtKB/TrEMBL entries.
UniProtKB also stores sequence clusters of 100, 90, and 50 percent identity in UniRef databases.
Finally, UniParc is the UniProt Archive and contains a complete set of the known set sequence, even those that are historically obsolete 
(\cite{uniprot2019}).
