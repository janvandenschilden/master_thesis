The website interface is a very convenient and user friendly way to retrieve information.
Sometimes it would be useful however to access and download this in automated way.
Imagine yourself collecting all the protein sequences that have an annotation of being in the cytoplasm for 20 different organisms of interest.
For each organism, a search query would have to be entered,
the download button has to be clicked and the FASTA format selected.
Then some more clicking to specify the download location, etc.
And this has to be repeated for every of the 20 organisms. 
Now imagine that halfway you notice that UniProtKB makes the distinction 
between the "cytoplasm" and "cytosol" annotation.
Both would serve the purpose of the research however, 
so to include "cytosol" proteins, you would have to start all over again.
Easier would be to just change the query parameter and automatically iterate through all the other steps.
Another reason why someone would like to automate the retrieval process is that the database is constantly growing.
More research is being done, and curators add manual annotations.
Moreover, automatic pipelines improve in accuracy, so also unreviewed annotations get better and more abundant.
Therefore it could be useful to reuse a search query and redo an analysis with a more up to date dataset.
UniProtKB provides programmatic access to its database in the form of a set of REST APIs.
Not only is there an API to programmatically perform a query search as in the example above,
there are also REST APIs for 
retrieving individual entries,
mapping IDs from one database to another,
format conversion, 
and more
(\cite{uniprot_api}).
