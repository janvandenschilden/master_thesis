% Rehearsal of main question
Examination of the difference in global biophysical features between cytoplasmic and periplasmic proteins in Gram-negative Bacteria seems to agree with the observations made by \cite{loos2019}.
Periplasmic proteins show an increased tendency towards the formation of coil and sheet structures,
but decreased propensity towards the formation of helical structures.
On average, periplasmic proteins display lower backbone and sidechain dynamics,
and a decreased tendency towards early folding.
This agrees with the hypothesis that periplasmic proteins have to delay the folding process to translocate to the periplasm.

% Main features observed (confirmation of other paper)
Region specific differences in biophysical features were examined by two complementary approaches.
The general approach including the periplasmic and cytoplasmic proteins of the class of Gammaproteobacteria.
In general, cytoplasmic proteins seem to display a region of increased flexibility and dynamics , and reduced propensity towards early folding near postion 10 to 20. 
In periplasmic proteins this region is extended by roughly 10 amino acids and has a slight shift towards the C-terminal end.
The first 10 amino acids however, 
display an increased tendency to early folding, 
an effect induced by the signal peptide.
Another interesting observation is that cytoplasmic proteins have an above average propensity to form a helix near position 25,
even while propensities to form a helix are averaged out over the large dataset in the rest of the protein.
In contrast, periplasmic proteins tend to have a decreased tendency towards helix formation starting from that same region until residue 70.
Periplasmic proteins have an increased tendency to form coil like structures and display increased backbone flexibility from position 20 to 70, 
except near position 33 where there is an increased tendency towards early folding and the formation of a sheet.

% local features observed by two complementary approaches
Similar differences were observed in the twin approach.
The effect of signal peptides on biophysical features near the N-terminal end was again observed,
most strongly in propensity towards helix formation and backbone dynamics.
As in previous approach, the region near position 33 is highlighted to display increased propensity towards early folding and decreased tendency to form helical structures in cytoplasmic proteins.
However, where in the general approach a tendency was seen to form sheet structures,
the twins showed by contrast an increased propensity to form helical structures.

% explanation
The reasons for these differences remain unclear.
Most likely, they are related to the delay of the folding process,
interactions with the translocation machinery,
or the formation of functional proteins in the periplasm.
It is possible that the high propensity towards early folding of the signal peptides,
and the increased propensity in the N-terminal region of the protein,
make very early folding foldons.
These foldons would than form interactions with other regions of the protein that do not occur in the fully folded structure,
thereby delaying the process.
The early folding region near position could possibly have a similar effect.
However, the position specificity seems to suggest that is forms some kind of interactions with translocation machinery.

% limitations
It should be noted that the tendencies described above don't need to be part of the same mechanism.
Also, there are very likely many biophysical differences that are position aspecific,
and can therefore not be elucidated by the analysis described above.
This research was purely based on sequential information and the predictions generated with these sequences.
However, a list of protein structures is provided as it was very straight forward to get these results with the UniProtKB pipeline that was designed for this master thesis. 
This list could be interesting to other research groups.

% UniRep prediction
Finally, the UniRep (\cite{alley2019}) representation was used to make classifiers that separates cytoplasmic and periplasmic proteins.
It showed that even without including signal peptides,
almost 95 percent of the proteins could be  correctly classified, 
which stresses the importance of the mature domain.
Though this was only an exploratory analysis, and a more rigorous approach should be performed to have u predictors.
The usage of UniRep vectors seems very promising.
Most importantly, they stress the fact there is a huge potential to use the undescribed protein sequences.
Including this information could improve the performance of predictors and make them applicable on a wider portion of the protein sequence space.
