% Part about UniProtKb
The amount of protein sequences in UniProtKB has been growing exponentially,
but experimental information of functionality or biophysical features is only available for about 0.3 percent of the protein sequence space.
% Relation between protein function and primary sequence
Local and distant interactions along the primary structure guide the folding process,
thereby determining structure , dynamics and function.
Essentially, all of the biophysical features are encoded within the amino acid sequence.
However, extracting those features has proven to be challenging and is still a topic of ongoing research.

% More specific about dynamics
Traditionally, protein function was thought to be a direct result of protein structure,
But in recent years it has become clear that structure provides only part of the picture.
Other biophysical features, such as the propensity toward dynamics, disorder, and early folding also seem to play a fundamental role for protein functionality.
A new generation of machine learning based predictors that only need a single protein sequence as input
are being developed at the Bio2Byte research lab at the Interuniversity Institute of Bioinformatics in Brussels.
Not only can these predictors predict biophysical features without relying on prior structural knowledge,
the predictions are also very fast,
making them scalable for computational high-throughput methods.

% What is unknown and what was researched
In this master thesis,
the biophysical features of cytoplasmic and periplasmic proteins were compared on an extensive Gram-negative Bacteria protein dataset.
Additionally, the effect of signal peptides on these features was examined.
This problem was approached from different directions.
A global approach showed that on average,
an increased tendency towards the formation coil and sheet structures,
but decreased propensity towards the formation of helical structures was observed in periplasmic proteins.
They also displayed lower backbone and sidechain dynamics,
and an decreased tendency towards early folding.
Two complementary positon specific approaches agreed on different features.
Signal peptides seem the promote the propensity towards helix formation and early folding in the N-terminal region of periplasmic proteins, 
but have an decreased propensity in the rest of the protein.
Additionally, a region near position 33 was observed with increased early folding propensity and decreased coil formation.
In a final approach, UniRep representations were utilized to make a predictor that can distinguish cytoplasmic and periplasmic proteins.

