% short rehearsal of the problem 
In order for periplasmic proteins to translocate they contain a signal peptide 
and seem to display different biophysical characteristics.
In this master thesis the difference in biophysical landscape between cytoplasmic 
and periplasmic proteins was examined by four complementary approaches.

% global approach
In the first approach,
a large dataset containing cytoplasmic and periplasmic proteins of the Gram-negative Gammaproteobacteria was extracted from UniProtKB,
and biophysical characteristics were compared.
It was examined whether periplasmic proteins are more likely to form certain secondary structures than cytoplasmic proteins.
Did they display an increased or decreased tendency toward sidechain and backbone dynamics,
and whether there were observations to indicate that periplasmic proteins fold more slowly than cytoplasmic ones.

% General comparison 
In the second approach, the same dataset and features were examined, 
but position within the protein was taken in into account.
It was examined if there was a general trend of features to appear in certain regions of the protein.
In addition, the biophysical characteristics of signal peptides was explored,
and there effect on the N-terminal regions of the mature domain of periplasmic proteins.

% twin comparison 
Evolution tends to tinker rather than invent.
This means new functionalities come into existence mostly by tweaking existing proteins, 
not by designing new ones from scratch. 
With this in mind,
there exist a whole set of proteins that homologues to each other, 
but one exist in the cytoplasm and on in the periplasm.
Such pairs will be further referred to as twins.
Interesting about these twins is that there structure and biophysical features are very similar,
and differences in characteristics are more likely related to them being in different compartments,
as that is their main difference.
Therefore, the third approach consisted of searching the UniProtKB databank for such twins and comparing there biophysical properties.
One advantage over previous approach is that twins can be directly aligned to each other,
making the differences more region specific.


% UniRep exploratory analysis
Finally, in the fourth approach it was examined how well UniRep representation could separate periplasmic proteins from cytoplasmic ones,
and whether the exclusion of the signal peptide had any effect on that.
The UniRep representation was constructed by letting a recurrent neural network explore the total of sequence space and build its own internal representation in a unsupervised way.
Therefore, it is unconstrained by mental models of the human mind, 
potentially improving classification.
This was a group project for the course "Techniques of Artificial Intelligence" in which I collaborated with 
Jessica Ody, 
Guillaume Buisson-Chavot, 
and Anthony Cnudde.
