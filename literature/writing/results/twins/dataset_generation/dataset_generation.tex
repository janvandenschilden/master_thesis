The twin dataset (see methods) is given in Table. \ref{table:twins}.

~\begin{longtable}[]{@{}lllllll@{}}
\toprule
& Cytoplasm uniref50 & cluster size & Periplasm uniref50 & cluster size \tabularnewline
\midrule
\endhead
0 & UniRef50\_Q2NVU4 & 124 & UniRef50\_A0A1X1D1X1 & 148 \tabularnewline
1 & UniRef50\_P83221 & 4189 & UniRef50\_O53021 & 2118 \tabularnewline
2 & UniRef50\_Q96VT4 & 1678 & UniRef50\_O59651 & 9906 \tabularnewline
3 & UniRef50\_P0A963 & 3836 & UniRef50\_P00805 & 2537 \tabularnewline
4 & UniRef50\_P0AAL2 & 1270 & UniRef50\_P0AAL4 & 492 \tabularnewline
5 & UniRef50\_P10902 & 6730 & UniRef50\_P0C278 & 160 \tabularnewline
6 & UniRef50\_P21517 & 4496 & UniRef50\_P25718 & 5447 \tabularnewline
7 & UniRef50\_O33732 & 29 & UniRef50\_P39185 & 6749 \tabularnewline
8 & UniRef50\_P44650 & 243 & UniRef50\_P44652 & 304 \tabularnewline
9 & UniRef50\_P0A9L3 & 1427 & UniRef50\_P45523 & 2225 \tabularnewline
10 & UniRef50\_P12994 & 1160 & UniRef50\_P77368 & 1614 \tabularnewline
11 & UniRef50\_Q8ZRT8 & 1014 & UniRef50\_Q32JZ5 & 878 \tabularnewline
12 & UniRef50\_Q72CB8 & 24 & UniRef50\_Q72EC8 & 17 \tabularnewline
13 & UniRef50\_A0A432R1Z2 & 206 & UniRef50\_Q7M827 & 379 \tabularnewline
14 & UniRef50\_P62602 & 2175 & UniRef50\_Q8XDH7 & 3943 \tabularnewline
15 & UniRef50\_Q32JZ5 & 878 & UniRef50\_Q8ZRT8 & 1014 \tabularnewline
\bottomrule
\caption{\textbf{Twins of UniRef50 clusters.}
Table shows twins of UniRef50 clusters with cluster size next to it.
Each twin of clusters contains at least one pair of twin proteins within the same organism.}
\label{table:twins}
~\end{longtable}

