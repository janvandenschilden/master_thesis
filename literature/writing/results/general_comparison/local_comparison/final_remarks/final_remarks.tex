It should be noted that comparing periplasmic to cytoplasmic protein on a large scale comes with its own set of advantages and limitations.
As the cytoplasmic dataset contained about 10 times more protein sequences than the periplasmic dataset,
the cytoplasmic curves are to be expected to appear flatter and smoother than the periplasmic curves.
As this analysis looked at variation in function of position,
only position specific tendencies can be observed.
This way it was observed that cytoplasmic proteins seemed to form a preference toward helix formation in the red region,
periplasmic proteins have an increased preference to form sheets in the yellow region,
and a general region of rigidity and early folding propensity followed by an relative flexible region of decreased early folding propensity near the beginning of a protein for both cytoplasmic and periplasmic proteins.
In periplasmic proteins, this flexible regions seems to be more extensive.
Only features that occur in a relative large part of the proteins will be revealed by this analysis,
which should make them more robust, but it has the downside that other features remain obscured.
However it is striking that even with this many proteins,
tendencies are observed without being averaged out.
Exact positions associated with the different regions should not be taken to strictly.
For example, the increased propensity towards helix formation observed at position 25  
would still be observed even if in reality this helix occurs more in distribution around position 25,
possibly with a quite high deviation.
It is also impossible to conclude from this analysis which tendencies occur together, which can but don't have to occur together and which are mutually exclusive.
Finally, 
features that rely more on a relative position to another feature rather than an absolute position will not be clarified by this analysis.
