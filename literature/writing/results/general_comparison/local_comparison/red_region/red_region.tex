The red region spans from residue 18 to 28.
Most noticeably, starting in this region up to the end of the blue region (residue 70),
there is a clear difference in propensity to form helices in favor for cytoplasmic proteins.
Similarly, backbone and sidechain dynamics appear to be more rigid for cytoplasmic than periplasmic proteins.
There is a higher propensity toward early folding for cytoplasmic than periplasmic proteins in the red domain,
with the difference being more pronounced when including the effect of signal peptides.
It is interesting to see that cytoplasmic proteins seem to have a general preference for helix formation in this region,
As in other parts of the protein, there seems to be more of a constant background preference as a result of averaging out over many proteins.
Combining this observation with previous one suggests that cytoplasmic proteins prefer on average a flexible region followed by an helix near the N-terminus.
Another possibility is that they prefer either an helix or a flexible region.
It is not possible to distinguish between these hypothesises with this analysis.
