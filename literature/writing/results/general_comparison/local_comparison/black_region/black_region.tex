
In the grey region, or the first N-terminal 11 residues,
cytoplasmic and periplasmic proteins tend to have a region of relative high backbone rigidity (higher backbone propensity values) , 
surrounded by more flexible regions (N-terminal end and green region).
However, in periplasmic proteins, this rigidity is mainly due to the effect of the signal peptide 
as mature domains without signal peptide are predicted to have an increased backbone flexibility (lower backbone propensity values).
Sidechain dynamics has a higher propensity toward rigidity in the cytoplasmic proteins than those of the periplasm.
Both the cytoplasmic and periplasmic mature domains seem to have a similar propensity toward early folding,
but the presence of a signal peptide has a significant increasing effect.
Within the grey region,
cytoplasmic proteins have a higher tendency to form helices than periplasmic mature domains (without signal peptides),
but a lower propensity than periplasmic proteins that still have a signal peptide.
Sheet formation is similar for both groups,
with the presence of a signal peptide having the effect of lowering this propensity.
Finally, coil formation seems to be more frequent in periplasm mature domains than cytoplasmic proteins,
the presence of a signal peptide significantly decreases the propensity to form a coil.
