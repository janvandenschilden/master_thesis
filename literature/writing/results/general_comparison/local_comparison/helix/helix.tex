~\begin{comment}
Fig. \ref{fig:local_helix}) shows that cytoplasmic proteins have a higher propensity to form a helix at the beginning and near residue position 25 (as shown by the bump in the blue curve).
More C-terminal position do not show this preference as is depicted by the relatively even curve.
Mature periplasmic domains on the other hand seem to have a lower propensity toward helix formation as is observed starting from position 0 going to position 75.
The signal peptide itself displays a high tendency towards this secondary structure however, 
and this effect seems to be propagated to the N-terminal part of the periplasmic proteins.

Why cytoplasmic proteins have an increased tendency to form helices at near position 0 and 25,
while this is decreased in the mature domains of the periplasm remains unclear.
These same general regions seem to have a relative high propensity toward early folding both in cytoplasmic and periplasmic proteins and (Fig. \ref{fig:local_earlyfolding),
though in periplasmic proteins they are shifted toward the C-terminal end.
This could suggest that these regions play an important role in the folding process,
decreasing the propensity toward helix formation possible has a delaying effect.
~\end{comment}

