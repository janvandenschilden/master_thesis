Fig.\ref{fig:local_features} (white regions) shows a clear signature of biophysical features for signal peptides.
In general, they have a higher propensity toward helix formation than toward the sheet or coil secondary structures.
However, for the most N-terminal part, the propensity towards forming a helix or forming a coil are very close to each other.
When reaching the region  close to position -20 (hydrophobic region), the propensity towards helix and sheet formation increases,
while the propensity towards coil formation decreases.

Within the C-region of the signal peptide, the propensity towards helix formation decreases again, 
but it is still the highest secondary structure propensity.
Generally, the C-terminal end of the signal peptides 
seems to have higher propensity to helix formation than the N-terminal end of the mature domain in periplasmic proteins.
Therefore, the signal peptide has positive effect towards forming a helix on the N-terminal end of the mature domain.
The C-region has a low propensity towards sheet formation,
and increasing propensity toward coil formation.
Also these tendencies propagates to the N-terminal part of the mature domain.  

In terms of dynamics, 
the H-region is the most rigid, both in backbone and sidechains.
In addition, it has high early folding propensity.
The C-region has a slight stabilizing effect on the sidechain dynamics of the N-terminal part of the mature domain,
and more significant stabilizing effect on the backbone dynamics.
The effect of the C-region on early folding propensity in the mature domain seems more complex.
While the N-terminal region of the mature domain increases in early folding propensity,
the rest of the protein seems to decrease.
One explanation for his behaviour is that the signal peptide can interact with the folding domains in the mature protein.
Because the signal peptide itself takes no part in the final protein,
all interactions with the signal peptide potentially delay the folding process.

In general, 
it seems that the signal peptides changes the secondary structure propensities, dynamical behaviour,
and early folding propensity of the N-terminal end of the mature domain.
These changed propensities increase the chance of forming interactions that are no part of the final fold,
thereby possible giving part of the mechanism behind the delayed folding observed in secreted proteins.
