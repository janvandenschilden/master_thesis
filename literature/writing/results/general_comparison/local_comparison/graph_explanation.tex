
For each residue of every protein, 
the following 6 biophysical features were predicted:
\textit{
helix,
sheet,
coil,
sidechain dynamics,
backbone dynamics},
and
\textit{early folding propensity}.
For every residue position,
the median, first and third quartile of every biophysical prediction was calculated.
This was done seperately for the cytoplasmic proteins,
periplasmic proteins,
and again for the periplasmic proteins but with the signal peptide cut off.
For example,
there are 30,632 cytoplasmic proteins in the dataset, thus 30,632 first residues. 
Each of these residues can have a different local environment, so different biophysical predictions.
The median and quartiles are then calculated for these 30,632 values and plotted on a graph.
This is repeated for all different residue positions, generating a continuous box-plot.
Similarly, medians and quartiles are calculated for the periplasmic proteins, one time with signal peptide and one time without.
Signal peptides are indicated by a negative residue position.
Because the proteins within this dataset have at most 50 percent sequence identity, 
protein specific tendencies will be averaged out.
Only very general position specific propensities that are true for most proteins will be detected.
Therefore, only the first 70 residues of the mature domain are shown in the graphs below.
Similarly, the general effect of the signal peptide on its local sequence environment is shown.
