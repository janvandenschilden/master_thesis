One question in the generation of a cytoplasmic and periplasmic protein dataset was on which taxonomic level to do the analysis. 
The most extensive approach would be use the UniProtKB annotation system to collect all the Bacterial proteins containing an  annotation of them being situated in the periplasm or cytoplasm.
However, since only Gram-negative Bacteria have a cytoplasm,
this would result in an uneven spread of proteins over the different taxa and introduce bias in the dataset.
Therefore,  only proteins from the class of Gammaproteobacteria were selected.
At the moment of writing, UniProtKB contained 28,310,322 protein sequences within this taxonomic group.
