The analysis in of this master thesis was purely sequence based.
In a next step it would be interesting to include pdb structures.
Therefore, the UniProtKB mapping API was used to map available structures to twin proteins (Table. \ref{table:structures}).

\newpage
\begin{longtable}[]{@{}lll@{}}
\toprule
& ACC & PDB\tabularnewline
\midrule
\endhead
0 & A0A156C5X3 & {[}5Z66, 5Z6H{]}\tabularnewline
1 & P00805 & {[}1HO3, 1IHD, 1JAZ, 1JJA, 1NNS, \tabularnewline
	& & 3ECA, 4ECA, 5MQ5, 6EOK, 6NX6, \tabularnewline
	& &6NX7, 6NX8, 6NX9, 6NXA, 6NXB, \tabularnewline
	& &6PA2, 6PA3, 6PA4, 6PA5, 6PA6, \tabularnewline
	& & 6PA8, 6PA9, 6PAA, 6PAB, 6PAC{]}\tabularnewline
2 & P0A962 & {[}2HIM, 2P2D, 2P2N, 6NXC, 6NXD{]}\tabularnewline
3 & P0AFL3 & {[}1CLH, 1J2A, 1V9T, 1VAI{]}\tabularnewline
4 & P12994 & {[}1FJJ, 1VI3{]}\tabularnewline
5 & P13482 & {[}2JF4, 2JG0, 2JJB, 2WYN{]}\tabularnewline
6 & P21517 & {[}5BN7{]}\tabularnewline
7 & P23869 & {[}1LOP, 2NUL, 2RS4{]}\tabularnewline
8 & P45523 & {[}1Q6H, 1Q6I, 1Q6U, 4QCC{]}\tabularnewline
9 & P77368 & {[}1FUX{]}\tabularnewline
10 & P83223 & {[}1D4C, 1D4D, 1D4E{]}\tabularnewline
11 & Q72EC8 & {[}2XVX, 2XVY, 2XVZ{]}\tabularnewline
12 & Q9Z4P0 & {[}1QO8{]}\tabularnewline
\bottomrule
\caption{\textbf{PDB structures available for twins.}
	\textbf{ACC.} Accession code used by UniProtKB.
	\textbf{PDB.} Accession code used by PDB.
}
\label{table:structures}
\end{longtable}

